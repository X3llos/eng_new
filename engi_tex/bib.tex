\begin{thebibliography}{999}

\bibitem{GamePhys} David M Bourg, \emph{Physics for Game Developers }, wydawnictwo O'Reilly Media, listopad 2001
\bibitem{Wzorce projektowe} Erich Gamma, Richard Helm, \emph{Wzorce projektowe. Elementy oprogramowania obiektowego wielokrotnego użytku}, wydawnictwo Helion, wrzesień 2010
\bibitem{Rotation} gafferongames.com, \emph{Rotation and Inertia Tensors}, \texttt{http://gafferongames.com/virtualgo/}, witryna internetowa, stan na 01 września 2014
\bibitem{nVidia_opencl} www.khronos.org, \emph{OpenCL 1.2 Reference Pages}, \texttt{http://www.khronos.org/registry/cl/sdk/1.2/docs/man/xhtml/}, witryna internetowa, stan na 01 września 2014
\bibitem{GLFW} GLFW 3.0 \emph{}, \texttt{http://www.glfw.org/docs/latest/} dokumentacja, stan na 1 września 2014
\bibitem{openGL} Kurs OpenGL \emph{}, \texttt{http://cpp0x.pl/kursy/Kurs-OpenGL-C++} dokumentacja, stan na 1 września 2014
\bibitem{nvidia} www.nvidia.pl \emph{}, \texttt{http://www.nvidia.pl/object/visual-computing-pl/}, witryna internetowa, stan na 1 września 2014
\bibitem{wiki1} en.wikipedia.org \emph{}, \texttt{http://en.wikipedia.org/wiki/High-level\_shader\_language/}, witryna internetowa, stan na 1 września 2014
\bibitem{SAT} Marek Zając, \emph{Algorytm SAT} \texttt{http://zajacmarek.com/wp-content/uploads/2012/12/Algorytm-SAT.pdf}, stan na 1 września 2014
\bibitem{math1} \emph{Iloczyn wektorowy} \texttt{http://www.math.edu.pl/iloczyn-wektorowy}, witryna internetowa, stan na 1 września 2014
\bibitem{wiki2} pl.wikipedia.org \emph{}, \texttt{http://pl.wikipedia.org/wiki/Lista\_moment\%C3\%B3w\_bezw\%C5\%82adno\%C5\%9Bci}, witryna internetowa, stan na 1 września 2014
\bibitem{wiki3} pl.wikipedia.org \emph{}, \texttt{http://pl.wikipedia.org/wiki/Zasady_dynamiki_Newtona}}, witryna internetowa, stan na 1 września 2014
\end{thebibliography}


