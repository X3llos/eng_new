\chapter{Wykorzystanie OpenCL w symulacji}

\section{Inicjalizacja OpenCL}
W części inicjalizującej odczytany zostaje plik kernela. Został on podzielony na cztery głowne sekcje - strukturę obietków, stałe, dodatkowe funkcje oraz samą funkcję kernela.

\subsection{Struktura obiektów}
W tej części zdefiniowana została struktura odzwierciedlająca dane, jakie posiadają obiekty użyte w symulacji. Aby dane przekazane z pamięci RAM do pamięci GPU mogły zostać poprawnie odczytane, kolejność definiowanych składowych obiektu musi być identyczna z kolejnością zdefiniowaną w części programu hosta.

\subsection{Stałe}
Ta część zawiera dane stałe, wykorzystywane przez funkcje pomocnicze. Zdefiniowane tu zostały takie dane jak wartość grawitacji czy opór powietrza.

\subsection{Funkcje pomocnicze}
Dzięki wzorowaniu się na standardzie C99 OpenCL pozwala na wykorzystanie funkcji. Umożliwia to znaczne zwiększenie czytelności oprogramowania oraz szybsze wprowadzenie ewentualnych zmian. W celu przyspieszenia wykonywania kernela argumenty funkcji są przekazywane przez wskaźnik, co niweluje konieczność tworzenia obiektów tymczasowych, a co za tym idzie dodatkowego czasu na alokowanie oraz zwalnianie duzych obszarów pamięci.
\newpage
\subsection{Kernel}
Tutaj zostają umieszczone funkcje, wykonywane w czasie działania aplikacji przez hosta. W przypadku stworzonej aplikacji użyte zostały dwie funkcje - aktualizująca obiekty, oraz sprawdzająca kolizje.

\section{Uruchomienie kernela przez hosta}
Funkcja kernela na samym początku pobiera identyfikator: \\
\centering\verb+unsigned int i = get_global_id(0);+\\\flushleft
Funkcja ta pobiera globalny identyfukator wątku. W przypadku stworzonej symulacji, operacje dla każdej bryły sztywnej biorącej udział w symulacji wykonywane są na oddzielnym wątku. Dzięki temu, identyfikator określa również daną bryłę, na której wykonujemy operacje. Odwoływanie się do elementów, na któych wykonywane są operacje, przypomina odwoływanie się do kolejnych elementów w tabeli. \\
Sam proces inicjalizacyjny ropoczynany jest od pobrania platform:\\
\centering\verb+ret = clGetPlatformIDs (1, &platform_id, &ret_num_platforms);+\\
\verb+ret = clGetDeviceIDs( platform_id, CL_DEVICE_TYPE_DEFAULT, 1, &device_id,+ \\
\verb+&ret_num_devices);+\\ \flushleft
W przypadku użytych urządzeń zostaną wykryte dwie plaftormy: karta NVIDIA GeForce GTX 760 oraz procesor Intel\textregistered Core\texttrademark i5-3570K. \\ 
Następnie utworzony zostaje kontekst obliczeniowy oraz tworzona jest kolejka zadań:\\
\centering\verb+context = clCreateContext(NULL, 1, &device_id, NULL, NULL, &ret);+\\
\verb+command_queue = clCreateCommandQueue(context, device_id, 0, &ret);+\\\flushleft
Kolejka służy jako interfejs do wysyłania do urządzenia zadań (tj. kopia danych do/z platformy, wysłanie kernela). \\
Kolejnym krokiem jest utworzenie obiektu programu, w którym używamy wcześniej stworzonego pliku kernela:\\
\centering\verb+program = clCreateProgramWithSource(context, 1, (const char **)&source_str,+\\
\verb+(const size_t *)&source_size, &ret);+\\\flushleft
Plik przekazywany jest z użyciem typu char*, zatem należy uprzednio wczytać plik do pamięci. \\
Po wykonaniu tych operacji należy zbudować program, używając stworzonego obiektu programu:\\
\centering\verb+ret = clBuildProgram(program, 1, &device_id, NULL, NULL, NULL);+\\\flushleft
Program budowany jest dla określonego urządzenia lub jeśli jako argument została podana wartość NULL, program jest budowany na wszystkich znalezionych urządzeniach. \\
Po zbudowaniu mozemy stworzyć obiekt kernela z programu:\\
\centering\verb+kernel = clCreateKernel(program, "updatePoints", &ret);+\\\flushleft
Jako argument podajemy nazwę funkcji z użytego pliku kernela. Stworzony obiekt kernela będzie można wysłać, po ustaleniu parametrów, za pomocą kolejki zadań do wykonania. \\
Następnie należy utworzyć zmienne typu cl\_mem, które będą służyły jako bufory wejścia/wyjścia w pamięci urządzenia np.:\\
\centering\verb+cl_a = clCreateBuffer(context, CL_MEM_READ_WRITE, sizeof(myBody)*numBoxes,+\\
\verb+NULL, &ret);+\\\flushleft
Każda zmienna tego typu może zostać oznaczona jako tylko do odczytu (CL\_MEM\_READ\_ONLY), tylko do zapisu (CL\_MEM\_WRITE\_ONLY) oraz z pełnym dostępem ( CL\_MEM\_READ\_WRITE). \\
Po utworzeniu danych należy przetransferować dane z pamięci hosta do pamięci urządzenia, gdyż kernel nie posiada bezpośredniego dostępu do danych umieszczonych w pamięci hosta. \\
\centering\verb+ret = clEnqueueWriteBuffer(command_queue, cl_a, CL_TRUE, 0,+\\
\verb+sizeof(myBody)*numBoxes, bodies, 0, NULL, NULL);+\\\flushleft
Jako, że kernel zostanie wysłany do kolejki zadań, należy odpowiednio ustawić parametry będące argumentami dla przesyłanej funkcji kernela, np.: \\
\centering\verb+ret = clSetKernelArg(kernel, 0, sizeof(cl_mem), &cl_a);+\\\flushleft
Gdy wszystkie etapy zostały wykonane, można przesłać obiekt kernela do kolejki zadań: \\
\centering\verb+ret = clEnqueueNDRangeKernel(command_queue, kernel, 1, NULL, &global_item_size,+\\
\verb+&global_item_size, 0, NULL, NULL);+\\\flushleft
Następnie należy zaczekać na wykonanie wszystkich obliczeń przez urządzenie:\\
\centering\verb+ret = clFinish(command_queue);+\\\flushleft
Jeśli ten krok zostanie pominięty a urządzenie nie skończy wykonywać kodu kernela, nasze dane wyjściowe będą błędne. \\
Gdy kernel zakończy prawidłowo pracę, należy przekopiować wynik z pamięci urządzenia do pamięci hosta: \\
\centering\verb+ret = clEnqueueReadBuffer(command_queue, cl_a, CL_TRUE, 0,+\\
\verb+sizeof(myBody)*numBoxes, bodies, 0, NULL, NULL);+\\\flushleft
Gdy dane zostaną przekopiowane, należy usunąć zainicjowane wcześniej obiekty OpenCL. \\
